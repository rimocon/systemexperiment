\documentclass{jarticle}
\usepackage[dvipdfmx]{graphicx}
\usepackage{here}
\usepackage{listings,jlisting}


\lstset{
  basicstyle={\ttfamily},
  identifierstyle={\small},
  commentstyle={\smallitshape},
  keywordstyle={\small\bfseries},
  ndkeywordstyle={\small},
  stringstyle={\small\ttfamily},
  frame={tb},
  breaklines=true,
  columns=[l]{fullflexible},
  numbers=left,
  xrightmargin=0zw,
  xleftmargin=3zw,
  numberstyle={\scriptsize},
  stepnumber=1,
  numbersep=1zw,
  lineskip=-0.5ex
}

\title{{システム実験}\\実験14回レポート}
\author{6119019056 山口力也}
\date{2019/07/26日提出}

\begin{document}
\maketitle
\section{演習}

\subsection{演習8.2.2}

\begin{table}[H]
\caption{カラーパターンの各色のRGB値}
	\begin{center}
		\begin{tabular}{|c|c|c|c|c|c|c|}\hline 
			& \multicolumn{3}{c|}{手動キャリブレーション後}&\multicolumn{3}{c|}{自動キャリブレーション後} \\ \hline 
			& Red値 & Green値 & Blue値 & Red値 & Green値 & Blue値 \\ \hline 
		黒 & 3 & 4 & 3 & 13 & 12 & 0  \\ \hline
		白 & 219 & 306 & 225 & 469 & 342 & 384  \\ \hline 
		赤 & 136 & 39 & 23 & 232 & 45 & 52  \\ \hline
		緑 & 49 & 150 & 62& 93 & 252 & 113  \\ \hline
		青 & 53 & 97 & 145 & 44 & 75 & 182  \\ \hline
		シアン & 41 & 89 & 74 & 138 & 261 & 291  \\ \hline
		マゼンタ & 124 & 74 & 94 & 255 & 144 & 206  \\ \hline
		イエロー & 187 & 259 & 64 & 326 & 303 & 117  \\ \hline
		\end{tabular}
	\end{center}
\label{table:enshu8-2-3} 
\end{table}

\subsection{演習8.2.3}

\begin{table}[H]
\caption{カラーパターンの各色に対するRGB値とその平均}
	\begin{center}
		\begin{tabular}{|c|c|c|c|c|c|c|c|c|c|c|c|c|}\hline 
		& \multicolumn{4}{c|}{Red値} & \multicolumn{4}{c|}{Green値} & \multicolumn{4}{c|}{Blue値} \\ \hline
			& 1 & 2 & 3 &平均& 1 & 2 & 3 &平均& 1 & 2 & 3 &平均 \\ \hline
		黒 & 0 & 6 & 4 & 3.3& 1 & 6 & 4 & 3.7& 0 & 5 & 3 & 2.7 \\ \hline
		白 &262&264&264&263.3&360&360&375&365&261&262&262&262.7 \\ \hline
		\end{tabular}
	\end{center}
\label{table:enshu8-2-5} 
\end{table}


\section{課題}

\subsection{課題8.2.2}

自動キャリブレーション時間の影響を調べるために,RGB値の最小,最大を決める時間を1秒と10秒でそれぞれ設定し,黒と白のRGB値を比較した.
以下表\ref{table:kadai8-2-2}に結果を示す.
\begin{table}[H]
\caption{カラーパターンの各色のRGB値}
	\begin{center}
		\begin{tabular}{|c|c|c|c|c|c|c|}\hline 
			& \multicolumn{3}{c|}{自動キャリブレーション(1s)}&\multicolumn{3}{c|}{自動キャリブレーション(10s)} \\ \hline 
			& Red値 & Green値 & Blue値& Red値 & Green値 & Blue値 \\ \hline
		黒 & 3 & 4 & 3 & 13 & 88 & 14 \\	 \hline
		白 & 73& 77& 14& 498& 506& 55\\ \hline 
		\end{tabular}
	\end{center}
\label{table:kadai8-2-2} 
\end{table}


\subsection{課題8.2.3}


\subsection{課題8.2.4}

\begin{table}[H]
\caption{カラーパターンの各色に対するRGB値とその平均}
	\begin{center}
		\begin{tabular}{|c|c|c|c|c|c|c|c|c|c|c|c|c|}\hline 
		& \multicolumn{4}{c|}{Red値} & \multicolumn{4}{c|}{Green値} & \multicolumn{4}{c|}{Blue値} \\ \hline
			& 1 & 2 & 3 &平均& 1 & 2 & 3 &平均& 1 & 2 & 3 &平均 \\ \hline
		黒 & 0 & 6 & 4 & 3.3& 1 & 6 & 4 & 3.7& 0 & 5 & 3 & 2.7 \\ \hline
		白 &262&264&264&263.3&360&360&375&365&261&262&262&262.7 \\ \hline
		赤 &157&152&150& 153 &75 &75 & 75& 75& 15& 15& 13& 14.3 \\ \hline
		緑 & 44& 54 &47& 3.3 &126&132&129& 3.7& 56&45& 47&  \\ \hline
		青 & 0 & 6 & 4 & 3.3& 1 & 6 & 4 & 3.7& 0 & 5 & 3 & 2.7 \\ \hline
 シアン & 0 & 6 & 4 & 3.3& 1 & 6 & 4 & 3.7& 0 & 5 & 3 & 2.7 \\ \hline
 イエロー& 0 & 6 & 4 & 3.3& 1 & 6 & 4 & 3.7& 0 & 5 & 3 & 2.7 \\ \hline
 マゼンタ& 0 & 6 & 4 & 3.3& 1 & 6 & 4 & 3.7& 0 & 5 & 3 & 2.7 \\ \hline
		\end{tabular}
	\end{center}
\label{table:enshu8-2-5} 
\end{table}


\end{document}
