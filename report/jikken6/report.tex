\documentclass{jarticle}
\usepackage[dvipdfmx]{graphicx}
\usepackage{here}
\usepackage{listings,jlisting}


\lstset{
  basicstyle={\ttfamily},
  identifierstyle={\small},
  commentstyle={\smallitshape},
  keywordstyle={\small\bfseries},
  ndkeywordstyle={\small},
  stringstyle={\small\ttfamily},
  frame={tb},
  breaklines=true,
  columns=[l]{fullflexible},
  numbers=left,
  xrightmargin=0zw,
  xleftmargin=3zw,
  numberstyle={\scriptsize},
  stepnumber=1,
  numbersep=1zw,
  lineskip=-0.5ex
}

\title{{システム実験}\\実験6回レポート}
\author{6119019056 山口力也}
\date{2019/05/28日提出}

\begin{document}
\maketitle

\section{ArduinoからProcessingへのデータ送信:通信方式1}
演習3.1.7の5で考察した内容を報告せよ.

\section{ArduinoからProcessingへのデータ送信:通信方式2}
演習3.1.8の5で考察した内容を報告せよ.

\section{繰り返し処理による描画(2Dバージョン)}
課題3.1.1で作成したスケッチ(draw\_latice\_2D)を報告せよ.

以下ソースコード\ref{code:kadai3-1-1}にスケッチを示す.

\begin{lstlisting}[caption = 課題3.1.1,label=code:kadai3-1-1][H]
int xn = 10; //x軸の格子
int yn = 10; //y軸の格子


int x,y; //x座標とy座標を定義

size(600,600); //ウィンドウサイズ600*600
background(255); //背景白

stroke(0); //線の色黒
for(int i =0; i<=yn; i++){ //格子の横線
  y =  i*height/yn; //線のy座標
  line(0,y,width,y); //線を描画
}

for(int i = 0; i<= xn; i++){ //格子の縦線
  x = i*width/xn; //線のx座標
  line(x,0,x,height); //線を描画
}

stroke(255,0,0);
for(int i_c =0; i_c<xn; i_c++){ //i_cはマス目のインデックス
  for(int i_r = 0; i_r < yn; i_r++){
    x = width/(2*xn) + i_c*width/xn; //30 + 60*i_c
    y = height/(2*yn) + i_r*height/yn; //30 + 60*i_r
    ellipse(x,y,width/xn,height/yn); //円を描画
  }
}
\end{lstlisting}

\section{動画の作成:移動する楕円(2Dバージョン)}
課題3.1.2で作成したスケッチ(draw\_latice\_2D)を報告せよ.

以下ソースコード\ref{code:kadai3-1-2}にスケッチを示す.

\begin{lstlisting}[caption = 課題3.1.2,label=code:kadai3-1-2][H]
int xn = 10; //分割数
int yn = 10; //分割数
int i_c; //楕円を描くマス目番号
int i_r; //楕円を描くマス目番号
int flag_upper_right = 0; //右上端フラグ
int flag_upper_left = 0; //左上端フラグ
int flag_down_right =0; //右下端フラグ
int flag_down_left = 0; //左下端フラグ

void setup(){
  size(600,600); //ウィンドウサイズ600*600
  frameRate(20); //フレームレート20
  i_c = 0; //列
  i_r = 0; //行
}
void draw()
{
  int x,y; //x座標とy座標
  background(255); //背景白
  stroke(0); //線の色黒
  for(int i = 0; i<= yn; i++){
    y =  i*height/yn; //線のy座標
    line(0,y,width,y);
  }
  for(int i = 0; i<= xn; i++){ //格子の縦線
    x = i*width/xn; //線のx座標
    line(x,0,x,height);
  }
  stroke(255,0,0);
  x = width/(2*xn) + i_c*width/xn; //30 + 60*i_c
  y = height/(2*yn) + i_r*height/yn; //30 + 60*i_r
  ellipse(x,y,width/xn,height/yn);
  //楕円を描くマス目番号の更新
  if ( i_c == 0 && i_r == 0){ //左上端の時
    flag_upper_left = 1;
    flag_upper_right = 0;
    flag_down_left = 0 ;
    flag_down_right = 0;
    println("upperleftnow");
  }
  if ( i_c == xn-1 && i_r == 0){ //右上端の時
    flag_upper_left = 0;
    flag_upper_right = 1;
    flag_down_left = 0 ;
    flag_down_right = 0;
    println("upperrightnow");
  }
  if ( i_c == 0 && i_r == yn-1){ //左下端の時

    flag_upper_left = 0;
    flag_upper_right = 0;
    flag_down_left = 1 ;
    flag_down_right = 0;
    println("downleftnow");
  }
  if ( i_c == xn-1 && i_r == yn-1){ //右下端の時

    flag_upper_left = 0;
    flag_upper_right = 0;
    flag_down_left = 0 ;
    flag_down_right = 1;
    println("downrightnow");
  }
  if(flag_upper_left == 1){
    i_c++;
  }
  if(flag_upper_right == 1){
    i_c--;
    i_r++;
  }
  if(flag_down_left == 1){
    i_c++;
  }
  if(flag_down_right == 1){
    i_c--;
    i_r--;
  }
  
}
\end{lstlisting}

\section{電圧変化の時間軸グラフによる可視化(点のプロット)}
課題3.1.3で作成したProcessingのスケッチを報告せよ.また,出力したウィンドウのスナップショットを報告せよ.
以下ソースコード\ref{code:kadai3-1-3}にスケッチを示す.

\begin{lstlisting}[caption = 課題3.1.3,label=code:kadai3-1-3][H]
import processing.serial.*; // Serial ライブラリを取り込む
Serial port; // Serial クラスのオブジェクトを宣言
int val,count,x,y; //変数宣言
void setup()
{
  size(800,300); // サイズ 800 × 300 のウィンドウ生成
  port = new Serial(this, "/dev/ttyACM0", 9600);//Serial クラスのインスタンス生成
  val = 0;
  count = 0;
  x = 0;
  y = 0;
  background(255); //背景白 
}
void draw()
{
  stroke(255,0,0); //赤色
  strokeWeight(5); //太さを5
  x = count; //countを代入
  y = (255 - val)*300/255; //val=0で300,val=255で0
  point(x,y); //点を描画
  if( count == 800){ //右端についたら
    count = 0; //初期化
    background(255); //背景をクリア 
  }
  println("R"); // 描画タイミング(確認用)
}
// シリアルポートにデータが到着するたびに呼び出される割り込み関数
void serialEvent(Serial p) { // p にはデータが到着したシリアルポートに対応するインスタンス(ここでは port )が代入される
  val = p.read(); // 受信バッファから 1 バイト読み込み
  println("<-");
  count++;
// データ受信タイミング(確認用)
}
\end{lstlisting}

また,以下図\ref{fig:kadai3-1-3}に出力したウィンドウのスナップショットを示す.

\begin{figure}[H]
\begin{center}
\includegraphics[width=7.0cm]{images/kadai3-1-3.png}
\caption{課題3.1.3の出力画像}
\label{fig:kadai3-1-3}
\end{center}
\end{figure}


\section{電圧変化の時間軸グラフによる可視化(折れ線)}
課題3.1.4で作成したProcessingのスケッチを報告せよ.また,出力したウィンドウのスナップショットを報告せよ.
以下ソースコード\ref{code:kadai3-1-4}にスケッチを示す.

\begin{lstlisting}[caption = 課題3.1.4,label=code:kadai3-1-4][H]
import processing.serial.*; // Serial ライブラリを取り込む
Serial port; // Serial クラスのオブジェクトを宣言
int val,count,new_x,new_y,old_x,old_y; //それぞれ値を定義
void setup()
{
  size(800,300); // サイズ 800 × 300 のウィンドウ生成
  port = new Serial(this, "/dev/ttyACM0", 9600);//Serial クラスのインスタンス生成
  val = 0; //Arduinoから送られてきたデータを格納する
  count = 0; //受信した回数(x座標)
  old_x = 0; //折れ線グラフ用の前回のxの値
  old_y = 0; //折れ線グラフ用の前回のyの値
  new_x = 0; //折れ線グラフ用の今のxの値
  new_y = 0; //折れ線グラフ用の今のyの値
  background(255); // 背景を白に設定
}
void draw()
{
  stroke(255,0,0); //線の色を赤に設定
  strokeWeight(5); //線の太さを"5"に設定
  new_x = count; //x座標の値はcountそのもの
  new_y = (255 - val)*300/255; //y座標の値はval=255のときy=0(一番上),val=0のときy=300(一番下)
  point(new_x,new_y); //点を描画
  strokeWeight(3); //線の太さを"3"に設定
  line(old_x,old_y,new_x,new_y); //折れ線用に線を描画
  old_x = new_x; //今のxの値を前回の値として格納
  old_y = new_y; //今のyの値を前回の値として格納
  if( count == 800){ //もし右端まできたら
    count = 0; //count初期化
    background(255); //背景を白に(クリア)
  }
  println("R"); // 描画タイミング(確認用)
}
// シリアルポートにデータが到着するたびに呼び出される割り込み関数
void serialEvent(Serial p) { // p にはデータが到着したシリアルポートに対応するインスタンス(ここでは port )が代入される
  val = p.read(); // 受信バッファから 1 バイト読み込み
  println("<-"); // データ受信タイミング(確認用)
  count++; //countを1増やす(x座標を1つ右にずらす)
}
\end{lstlisting}
また,以下図\ref{fig:kadai3-1-4}に出力したウィンドウのスナップショットを示す.

\begin{figure}[H]
\begin{center}
\includegraphics[width=7.0cm]{images/kadai3-1-4.png}
\caption{課題3.1.4の出力画像}
\label{fig:kadai3-1-4}
\end{center}
\end{figure}

\section{電圧変化のメーター表示}
課題3.1.5で作成したProcessingのスケッチを報告せよ.また,出力したウィンドウのスナップショットを報告せよ.
以下ソースコード\ref{code:kadai3-1-5}にスケッチを示す.

\begin{lstlisting}[caption = 課題3.1.5,label=code:kadai3-1-5][H]
import processing.serial.*; // Serial ライブラリを取り込む
Serial port; // Serial クラスのオブジェクトを宣言
float rad,x,y; //ラジアン,x,yをそれぞれ定義
int val; //Arduinoから送られてくるデータ格納用

void setup()
{
  size(500,500); // サイズ 500 × 500 のウィンドウ生成
  port = new Serial(this, "/dev/ttyACM0", 9600);//Serial クラスのインスタンス生成

}
void draw()
{
  background(255); //背景を白に設定
  stroke(0,0,0); //線を黒に設定
  ellipse(200,200,200,200); //円を描画
  stroke(255,0,0); //線を赤に設定
  strokeWeight(5); //線の太さを"5"に設定
  rad = 2*PI*val/255; //受信したデータをラジアンに変換(255で2π,0で0)
  x = 200 + 100*cos(rad - PI/2); //x座標の値(中心200から距離cos(rad - π/2)
  y = 200 + 100*sin(rad - PI/2); //y座標の値(中心200から距離sin(rad - π/2)
  line(200,200,x,y); //メータの線を描画
  println("R"); // 描画タイミング(確認用)
}
// シリアルポートにデータが到着するたびに呼び出される割り込み関数
void serialEvent(Serial p) { // p にはデータが到着したシリアルポートに対応するインスタンス(ここでは port )が代入される
  val = p.read(); // 受信バッファから 1 バイト読み込み
  println("<-"); // データ受信タイミング(確認用)
}
\end{lstlisting}
また,以下図\ref{fig:kadai3-1-5}に出力したウィンドウのスナップショットを示す.

\begin{figure}[H]
\begin{center}
\includegraphics[width=7.0cm]{images/kadai3-1-5.png}
\caption{課題3.1.5の出力画像}
\label{fig:kadai3-1-5}
\end{center}
\end{figure}


\section{発展課題3.1.1 複数データの送信}
発展課題3.1.1で作成したArduinoとProcessingのスケッチを報告せよ.
以下ソースコード\ref{code:hatten3-1-1}にスケッチを示す.

\begin{lstlisting}[caption = 発展課題3.1.1,label=code:hatten3-1-1][H]


\end{lstlisting}

また,以下図\ref{fig:hatten3-1-3}に出力したウィンドウのスナップショットを示す.

\begin{figure}[H]
\begin{center}
\includegraphics[width=7.0cm]{images/hatten3-1-3.png}
\caption{発展課題3.1.3の出力画像}
\label{fig:hatten3-1-3}
\end{center}
\end{figure}


\section{発展課題3.1.2 バーコレーションシミュレーション}
