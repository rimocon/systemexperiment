\documentclass{jarticle}
\usepackage[dvipdfmx]{graphicx}
\usepackage{here}
\usepackage{listings,jlisting}
\usepackage{amsmath}

\lstset{
  basicstyle={\ttfamily},
  identifierstyle={\small},
  commentstyle={\smallitshape},
  keywordstyle={\small\bfseries},
  ndkeywordstyle={\small},
  stringstyle={\small\ttfamily},
  frame={tb},
  breaklines=true,
  columns=[l]{fullflexible},
  numbers=left,
  xrightmargin=0zw,
  xleftmargin=3zw,
  numberstyle={\scriptsize},
  stepnumber=1,
  numbersep=1zw,
  lineskip=-0.5ex
}

\title{{システム実験}\\実験第13回レポート}
\author{6119019056 山口力也}
\date{2019/07/19日提出}

\begin{document}
\maketitle
\section{目的}
本実験では以下の項目を目標とした.
\begin{itemize}
\item カラーセンシング技術を理解し,応用できる.
\item カラーセンサのキャリブレーションを理解し,応用できる.
\item $I^2-C$による通信方法を理解し,状況に合わせて通信プログラムの設定を変更できる.
\item 色空間の変換方法を理解し,色情報をProcessingで可視化できる.
\end{itemize}

\section{演習}
\subsection{演習8.1.1}
カラーセンサの色計測を行うために以下図\ref{fig:enshu8-1-1}に示す回路を構成した.

\begin{figure}[H]
\begin{center}
\includegraphics[width=7.0cm]{images/enshu8-1-1.png}
\caption{演習8.1.1の回路}
\label{fig:enshu8-1-1}
\end{center}
\end{figure}

\subsection{演習8.1.2}
Arduinoで,$I^2-C$による通信を実行するためのWire関数の使い方について理解した.

以下ソースコード\ref{code:enshu8-1-2-a}に作成したプログラムのソースコードを示す.

\lstinputlisting[caption = 演習8.1.2(Arduino),label=code:enshu8-1-2-a]{enshu8-1-2.ino}

\subsection{演習8.1.3}
演習8.1.2で取得したカラーセンサの値をProcessingで可視化した.

以下ソースコード\ref{code:enshu8-1-3-p}に作成したプログラムのソースコードを示す.

\lstinputlisting[caption = 演習8.1.3(Processing),label=code:enshu8-1-3-p]{enshu8_1_3.pde}

\subsection{演習8.1.4}
フルカラーLEDにより基本色であるRGBを再現した.
以下図\ref{fig:enshu8-1-4}にブレッドボード上の構成図を示す.

\begin{figure}[H]
\begin{center}
\includegraphics[width=7.0cm]{images/enshu8-1-4.png}
\caption{演習8.1.4の配線図}
\label{fig:enshu8-1-4}
\end{center}
\end{figure}

\subsection{演習8.1.5}
カラーセンサによりフルカラーLEDの値を読み出しシリアルモニタに表示するプログラムを作成した.ただし同期タイミングを合わせるため,タイマ割り込みを用いた.
以下ソースコード\ref{code:enshu8-1-5-a}に作成したプログラムのソースコードを示す.

\lstinputlisting[caption = 演習8.1.5(Arduino),label=code:enshu8-1-5-a]{enshu8-1-5-2.ino}

\subsection{演習8.1.6}
RGB値からxy値に変換するためのプログラムを作成した.
ここで変換行列に
\begin{equation}
A = 
\begin{pmatrix}
-0.142 & 1.549 & -0.956 \\
-0.334 & 1.578 & -0.731 \\
-0.682 & 0.770 & 0.563 
\end{pmatrix}
\end{equation}

を用いた.
以下ソースコード\ref{code:enshu8-1-6-a}に作成したプログラムのソースコードを示す.

\lstinputlisting[caption = 演習8.1.6(Arduino),label=code:enshu8-1-6-a]{enshu8-1-6.ino}

\section{課題}
\subsection{課題8.1.1}
演習8.1.3で作成したプログラムを元に,Clear値についても可視化した.\\
また,カラーセンサの値を読み取る通信速度についても検討した.これについてはタイマー割り込みを用いて同期させることで改善した.

以下ソースコード\ref{code:kadai8-1-1-a},\ref{code:kadai8-1-1-p}にそれぞれ作成したプログラムのソースコードを示す.

\lstinputlisting[caption = 課題8.1.1(Arduino),label=code:kadai8-1-1-a]{kadai8-1-1.ino}

\lstinputlisting[caption = 課題8.1.1(Processing),label=code:kadai8-1-1-p]{kadai8_1_1.pde}


\subsection{課題8.1.2}
演習8.1.5のスケッチを元に,Green,Blueを加えまた,シアン,マゼンタ,イエローを表現するプログラムを完成させた.

以下ソースコード\ref{code:kadai8-1-2-a}に作成したプログラムのソースコードを示す.

\lstinputlisting[caption = 課題8.1.2(Arduino),label=code:kadai8-1-2-a]{kadai8-1-2.ino}

結果としては,青と赤でマゼンタ,青と緑でシアン,赤と緑でイエローがそれぞれフルカラーLEDの出力として得られた.

\subsection{課題8.1.3}
取得したカラーセンサのRGB値をXYZ値に変換後,xy色度図上に投影した.ただしオートキャリブレーションを行い,あらかじめキャリブレーションはするものとした.


以下ソースコード\ref{code:kadai8-1-3-a},\ref{code:kadai8-1-3-p}にそれぞれ作成したプログラムのソースコードを示す.

\lstinputlisting[caption = 課題8.1.3(Arduino),label=code:kadai8-1-3-a]{kadai8-1-3_kaizen.ino}

\lstinputlisting[caption = 課題8.1.3(Processing),label=code:kadai8-1-3-p]{kadai8_1_3.pde}

また,以下図\ref{fig:kadai8-1-3}にプログラムの実行結果の図を示す.

\begin{figure}[H]
\begin{center}
\includegraphics[width=7.0cm]{images/kadai8-1-3.png}
\caption{課題8.1.3の実行結果}
\label{fig:kadai8-1-3}
\end{center}
\end{figure}

\subsection{課題8.1.4}
取得したカラーセンサのRGB値をXYZ値に変換後,xy色度図上に投影した.ただしオートキャリブレーションを行わず,あらかじめカラーセンサの値を用いてマニュアルキャリブレーションするものとした.


以下ソースコード\ref{code:kadai8-1-4-a},\ref{code:kadai8-1-4-p}にそれぞれ作成したプログラムのソースコードを示す.

\lstinputlisting[caption = 課題8.1.4(Arduino),label=code:kadai8-1-4-a]{kadai8-1-4.ino}

\lstinputlisting[caption = 課題8.1.4(Processing),label=code:kadai8-1-3-p]{kadai8_1_4.pde}

また,以下図\ref{fig:kadai8-1-4}にプログラムの実行結果の図を示す.

\begin{figure}[H]
\begin{center}
\includegraphics[width=7.0cm]{images/kadai8-1-4.png}
\caption{課題8.1.4の実行結果}
\label{fig:kadai8-1-4}
\end{center}
\end{figure}


\end{document}
